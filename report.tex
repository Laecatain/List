\documentclass[UTF8]{ctexart}
\usepackage{geometry, CJKutf8}
\geometry{margin=1.5cm, vmargin={0pt,1cm}}
\setlength{\topmargin}{-1cm}
\setlength{\paperheight}{29.7cm}
\setlength{\textheight}{25.3cm}

% useful packages.
\usepackage{amsfonts}
\usepackage{amsmath}
\usepackage{amssymb}
\usepackage{amsthm}
\usepackage{enumerate}
\usepackage{graphicx}
\usepackage{multicol}
\usepackage{fancyhdr}
\usepackage{layout}
\usepackage{listings}
\usepackage{float, caption}

\lstset{
    basicstyle=\ttfamily, basewidth=0.5em
}

% some common command
\newcommand{\dif}{\mathrm{d}}
\newcommand{\avg}[1]{\left\langle #1 \right\rangle}
\newcommand{\difFrac}[2]{\frac{\dif #1}{\dif #2}}
\newcommand{\pdfFrac}[2]{\frac{\partial #1}{\partial #2}}
\newcommand{\OFL}{\mathrm{OFL}}
\newcommand{\UFL}{\mathrm{UFL}}
\newcommand{\fl}{\mathrm{fl}}
\newcommand{\op}{\odot}
\newcommand{\Eabs}{E_{\mathrm{abs}}}
\newcommand{\Erel}{E_{\mathrm{rel}}}

\begin{document}

\pagestyle{fancy}
\fancyhead{}
\lhead{陈俊翰, 学号3230102940}
\chead{数据结构与算法第四次作业}
\rhead{2024年10月17日}

\section{测试程序的设计思路}

在设计 `List.cpp` 中的测试时,我采取了以下步骤:

首先,创建了一个空的链表对象 `List<int> lst`。接着,我试图在一个空列表上调用 `front()` 和 `back()` 方法。

然后,我向列表中添加了多个整数元素,并测试了 `push\_back()` 方法的功能。在添加元素之后,我调用了 `pop\_back()` 方法来移除列表的最后一个元素,并检查了列表的状态是否符合预期。

我还测试了迭代器的前后置自增和自减操作,以确保迭代器能够在列表中正确地移动。此外,我还测试了构造函数和移动语义,以验证列表对象在初始化和移动时的行为。

接着,我使用 `find()` 方法查找列表中的特定元素和不存在元素,并尝试删除列表中不存在的元素。之后,我尝试使用已删除元素的迭代器。

最后,我使用了 `erase()` 方法来删除列表中的元素,并验证了删除操作是否正确地更新了列表的状态。


\section{测试的结果}

测试结果:

1. 当列表为空时,尝试调用 `front()` 和 `back()` 方法,输出如下:
\begin{verbatim}
First element of empty list: 0
Last element of empty list: 0
\end{verbatim}

2. 向列表中添加了 10 个元素,并随后调用了 `pop\_back()` 方法删除了最后一个元素,输出如下:
\begin{verbatim}
Elements after adding 10 elements:
0 1 2 3 4 5 6 7 8 9 
Elements after popping last element:
0 1 2 3 4 5 6 7 8
\end{verbatim}

3. 测试了迭代器的前后置自增和自减操作,迭代器能够在列表中正确地移动,输出如下:
\begin{verbatim}
Element before end: 8 
\end{verbatim}

4. 使用构造函数初始化了一个列表,并通过移动语义将其移动到另一个列表中,输出如下:
\begin{verbatim}
Initialized list: 1 2 3 4 5 
Moved list: 1 2 3 4 5 
Moved assigned list: 1 2 3 4 5 
\end{verbatim}

5. 测试了 `find()` 方法的功能,输出如下:
\begin{verbatim}
Found element 5 at index: 5
Element -1 not found.
\end{verbatim}

6。删除不存在的元素,并使用已删除元素的迭代器,输出如下:
\begin{verbatim}
Element 15 not found in the list.
Value at deleted iterator: 20641678
\end{verbatim}

7. 使用 `erase()` 方法删除了列表中的元素,验证了删除操作是否正确地更新了列表的状态,输出如下:
\begin{verbatim}
list after being erased:
0 2 3 4 5 6 7 
\end{verbatim}

此外,我还使用 `valgrind` 工具对程序进行了内存泄漏检测,结果显示没有内存泄漏的问题。

\section{(可选)bug报告}

我发现了一个 bug,触发条件如下:

\begin{enumerate}  
    \item 首先,创建一个空的双向链表 `List<int> lst`。  
    \item 然后,尝试调用 `lst.front()` 和 `lst.back()` 方法。  
    \item 此时发现,程序返回了默认构造的 `int` 值 `0`,而不是抛出异常或返回特定的错误指示。  
\end{enumerate}  

我发现了另一个 bug,触发条件如下:

\begin{enumerate}  
    \item 创建一个迭代器 \texttt{List<int>::iterator it2 = lst.begin()}。
    \item 将迭代器 \texttt{it2} 向前移动一位 \texttt{++it2}。
    \item 调用 \texttt{lst.erase(it2)} 删除迭代器指向的元素。
    \item 尝试访问已删除元素的迭代器 \texttt{*it2}。
    \item 此时发现,尝试访问已删除元素的迭代器 \texttt{*it2} 导致了未定义行为。
\end{enumerate}  

据我分析,它们出现的原因是:

\begin{enumerate}
    \item 访问空列表的 `front()` 和 `back()` 方法时,程序没有抛出异常或返回特定的错误指示,而是返回了默认构造的 `int` 值 `0`。这可能导致后续逻辑处理时产生误解。
    
    \item 在删除元素后,迭代器没有正确更新,导致后续操作使用了无效的迭代器。具体来说,在删除元素后,迭代器 `it2` 仍然指向已被删除的位置,因此访问 `*it2` 导致了未定义行为。

\end{enumerate}

\end{document}

